There are around 300 million cattle worldwide. However the systems managing and tracking them varies significantly. Huge discrepancies between the procedures implement in developed and developing countries.

The EU has implemented a system of complete traceability from birth to death:
\begin{itemize}
	\item All cattle must have a unique identifier composed of the country code, herd mark, check digit and individual number.
	\item All cattle are tagged shortly after birth (one tag with the unique identifier on each ear). Newer tags contain RFID tags containing additional information about the cattle. 
	\item All cattle must have a bovine passport containing their date of birth, their breed and information on their genetic lineage.
	\item Registers of cattle herds must be maintained and regularly inspected.
	\item All cattle movements (both national and international) must be maintained in a digital database.
\end{itemize}

Additionally, ever since a BSE (mad cow disease) outbreak in 1997, each step in the food chain is scrutinized. From cattle to milk or burger, a database of all relevant information (health, origin, etc.) is maintained. This means any issues can be traced back to a farm, a particular cow, and batch of feed they were eating. These measures proved to be particularly useful during the horse meat scandal. Experts were able to trace back the problem from Findus products in the UK, through firms in France and Holland, back to two Romanian slaughterhouses. However, even with the stringent regulations aforementioned, some farmers still try to cheat the system for financial gain. Currently, there is no method to uniquely identify an animal without physical tags and thus prevent cattle identity usurpation.

On the other hand, in developing countries, many of these systems are lacking. Unlike in the developed world, agricultural workers make up between 50\% and 90\% of the population’s working force. Additionally, herds are usually composed of only a handful of cattle in contrast with large number of animal on industrialised farms. Due to the lack of capital investment and economies of scale, local authorities are often unable to implement robust regulation and enforce strict cattle tagging and tracking protocols. Currently, most official systems are quite ineffective and inefficient.
\begin{itemize}
	\item Governmental records are often handwritten note which are not standardised between regions.
	\item Farmers often do not use tags but rather cut unique notches of the cattle’s ears or even brand them to identify them.
	\item Police officers are quite lax and are sometimes bribed to ignore effractions of the law.
\end{itemize}

Nonetheless, Africa on the whole is undergoing a technological revolution. In terms of communication infrastructure, many countries are bypassing developing landlines and have already implemented robust 3G and 4G networks. Many people already own a smartphone. African countries are keen and have been trying to implement proper book keeping to open up access to valuable foreign export markets like the EU.

Thus, there is a true need for a system that can provide reliable identification. It could vastly improve the economic circumstances of many farmers and could help prevent public disease outbreaks. Also the current methods used for tagging infringe certain animal rights.  CowHub is exactly that - an alternative solution to the current protocols that have been widely used in the cattle industry. It has been designed to solve the following undesirable characteristics and issues of its predecessors:

\begin{itemize}
	\item \textbf{Inhumane}: tag needs to be physically installed (pierced) into the ear of the cattle
	\item \textbf{Inefficient}: registration and the production of tags is time-consuming
	\item \textbf{Unsafe}: tags, and other physical additions, can be easily removed, removing a farmer's only record of ownership of their cattle
	\item \textbf{Obsolete}: with contention from the public, the system has not been updated or reviewed in a significant period of time
\end{itemize}

CowHub introduces a new and modern way to get around the aforementioned issues. CowHub is able to identify a cattle by using its biometric features\footnote{The muzzle as of now.}. This approach is non-intrusive as as does not relly on any physically attachment such as a tag to recognize cattle. Also, the biometrics features used are unique and do not change over time. Biometric recognition techniques have been widely used to prevent fraud, enhance security, and curtail identity theft\cite{biometrics}. Thus, the identity and integrity of each individual is guaranteed to be preserved.
