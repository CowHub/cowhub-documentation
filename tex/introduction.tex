CowHub is an alternative solution to the current tagging system (the ``system'') that has been widely used in the cattle industry. CowHub ought to replace the system which are thought to be 
\begin{itemize}
	\item inhumane, as the tag needs to be physically installed (pierced) into the ear of the cattle;
	\item inefficient, as the registration and the production of the tags take a lot of time and work;
	\item unsafe, as the tag can be easily removed, leaving no means for a cattle to be identified;
	\item obsolete, as the system has been around for some time and is not up to the modern system standard; 
\end{itemize}

CowHub introduces a new and modern way to get around the aforementioned issues. CowHub uses biometric features\footnote{The muzzle as of now.} to identify a cattle, meaning that there needn't be any physical attachment to be done to any cattle, and as the biometrics are unique, measurable physical characteristics that are used widely in the industry for object identification\cite{biometrics}. The registration, analysation and the identification are carried out on the cloud, and the processes have been considerately accelerated through the introduction of the mobile (including Android and iOS) and web front-ends.

CowHub is a relatively big project and unlike other projects of smaller scale, care must be taken in terms of project provision and management to allow for a steadier development flow - this has been enabled for us through the use of various managerial tools. 

Furthermore, CowHub itself has been disintegrated into a number of independent components that communicate with each other in a secured environment\footnote{We have created Virtual Private Cloud networks for this reason.} through standardised methods (the REST API, for example) to allow for user extension\footnote{User extension of the system is encouraged as we believe in the idea of openness of a system.}, should one believes that it better suits his/her needs.

CowHub is designed to be efficient. It is suggested, from the purposed structure, that there is no correlation between the computation time and the size of the database even though as of now, the computational unit in essence is running a process analogous to MapReduce. Details can be found in the Design and Implementation section.