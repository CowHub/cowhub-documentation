%------------------------------------------------------------------------------
% No implementation, software engineering details, or project management
%------------------------------------------------------------------------------

%------------------------------------------------------------------------------

\begin{subsection}{The Elevator Pitch}
  Whilst advances in technology have revolutionised other industries and parts of agriculture, bringing with it radically improved methods and cost efficiencies, the way in which we identify and mark cattle is practically archaic.
\end{subsection}
 
%------------------------------------------------------------------------------

\begin{subsection}{Project Description}
  This project will focus on providing a technology-based answer to this problem by providing a user friendly platform, alleviating all associated pain for cattle, whilst offering benefits that only a scalable, centralised system is able to offer, as above.
    
  There are three key objectives for this project:
  
  \begin{itemize}
  	\item To offer a \textbf{harmless, physically non-destructive alternative} to current methods used to identify cattle
  	\item Provide a service that \textbf{allows the identification of cattle}
  	\item Collect data on cattle from farmers to be able to \textbf{build and use a widespread catalogue of information on and concerning cattle and their movements}
  \end{itemize}
\end{subsection}

%------------------------------------------------------------------------------

\begin{subsection}{What is the need for the project?}
  The process of marking cattle can be incredibly painful, regardless of the age of the cattle. Clearly, being subject to this pain, and the witnessing of the pain of a calf by a mother, is a point of controversy for animal rights activists. Members of the public feel so strongly about the issue that they have protested by branding themselves \cite{theguardian1}. 
  
  Whilst branding itself is no longer carried out in the United Kingdom, methods including piercing a cattle's ears are still in use today and are without competition currently. 

  \begin{figure}[H]
  	\centering
    \includegraphics[width=0.5\textwidth]{images/cattle-with-ear-tag.jpg}
  	\caption[Cattle ear tagging]{
      Illustration of a cattle's head, demonstrating the position and rough size of cattle tags from the front. \cite{wikihow1}
  	}
  \end{figure}
  
  Hence, this project will provide a user friendly platform to easily register and manage the information of a cattle without having to hurt it in any way. In fact, a simple picture as well as the usual informations to register a cattle would needed to identify a cattle.

\end{subsection}
