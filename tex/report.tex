%%%%%%%%%%%%%%%%%%%%%%%%%%%%%%%%%%%%%%%%%
% Structured General Purpose Assignment
% LaTeX Template
%
% This template has been downloaded from:
% http://www.latextemplates.com
%
% Original author:
% Ted Pavlic (http://www.tedpavlic.com)
%
% Note:
% The \lipsum[#] commands throughout this template generate dummy text
% to fill the template out. These commands should all be removed when
% writing assignment content.
%
%%%%%%%%%%%%%%%%%%%%%%%%%%%%%%%%%%%%%%%%%

%----------------------------------------------------------------------------------------
%	PACKAGES AND OTHER DOCUMENT CONFIGURATIONS
%----------------------------------------------------------------------------------------

\documentclass[geometry:a4paper]{article}

\usepackage{fancyhdr} % Required for custom headers
\usepackage{lastpage} % Required to determine the last page for the footer
\usepackage{extramarks} % Required for headers and footers
\usepackage{graphicx} % Required to insert images
\usepackage{lipsum} % Used for inserting dummy 'Lorem ipsum' text into the template

% Margins
\topmargin=-0.45in
\evensidemargin=0in
\oddsidemargin=0in
\textwidth=6.5in
\textheight=9.0in
\headsep=0.25in

\linespread{1.1} % Line spacing

% Set up the header and footer
\pagestyle{fancy}
\lhead{\docAuthorName} % Top left header
\rhead{\firstxmark} % Top right header
\lfoot{\lastxmark} % Bottom left footer
\cfoot{} % Bottom center footer
\rfoot{Page\ \thepage\ of\ \pageref{LastPage}} % Bottom right footer
\renewcommand\headrulewidth{0.4pt} % Size of the header rule
\renewcommand\footrulewidth{0.4pt} % Size of the footer rule

\setlength\parindent{0pt} % Removes all indentation from paragraphs

%----------------------------------------------------------------------------------------
%	DOCUMENT STRUCTURE COMMANDS
%	Skip this unless you know what you're doing
%----------------------------------------------------------------------------------------

% Header and footer for when a page split occurs within a problem environment
\newcommand{\enterProblemHeader}[1]{
\nobreak\extramarks{#1}{#1 continued on next page\ldots}\nobreak
\nobreak\extramarks{#1 (continued)}{#1 continued on next page\ldots}\nobreak
}

% Header and footer for when a page split occurs between problem environments
\newcommand{\exitProblemHeader}[1]{
\nobreak\extramarks{#1 (continued)}{#1 continued on next page\ldots}\nobreak
\nobreak\extramarks{#1}{}\nobreak
}

\setcounter{secnumdepth}{0} % Removes default section numbers
\newcounter{docProblemCounter} % Creates a counter to keep track of the number of problems

\newcommand{\docProblemName}{}
\newenvironment{docProblem}[1][Problem \arabic{docProblemCounter}]{ % Makes a new environment called docProblem which takes 1 argument (custom name) but the default is "Problem #"
\stepcounter{docProblemCounter} % Increase counter for number of problems
\renewcommand{\docProblemName}{#1} % Assign \docProblemName the name of the problem
\section{\docProblemName} % Make a section in the document with the custom problem count
\enterProblemHeader{\docProblemName} % Header and footer within the environment
}{
\exitProblemHeader{\docProblemName} % Header and footer after the environment
}

\newcommand{\problemAnswer}[1]{ % Defines the problem answer command with the content as the only argument
\noindent\framebox[\columnwidth][c]{\begin{minipage}{0.98\columnwidth}#1\end{minipage}} % Makes the box around the problem answer and puts the content inside
}

\newcommand{\docSectionName}{}
\newenvironment{docSection}[1]{ % New environment for sections within doc problems, takes 1 argument - the name of the section
\renewcommand{\docSectionName}{#1} % Assign \docSectionName to the name of the section from the environment argument
\subsection{\docSectionName} % Make a subsection with the custom name of the subsection
\enterProblemHeader{\docProblemName\ [\docSectionName]} % Header and footer within the environment
}{
\enterProblemHeader{\docProblemName} % Header and footer after the environment
}

%----------------------------------------------------------------------------------------
%	NAME AND CLASS SECTION
%----------------------------------------------------------------------------------------

\newcommand{\docTitle}{Tracing and identifying cattle} % Assignment title
\newcommand{\docDate}{Monday, January 9th 2017} % Due date
\newcommand{\docAuthorNameFirst}{Frederick Lindsey, Thomas Szyszko}
\newcommand{\docAuthorNameSecond}{Karim Nahas, Hongjiang Liu, Gregoire Yharrassarry}

%----------------------------------------------------------------------------------------
%	TITLE PAGE
%----------------------------------------------------------------------------------------

\title{
\vspace{2in}
\textmd{\textbf{\hmwkClass}}\\
\textmd{\textbf{\docTitle}}\\
\textmd{{\docDate}}\\
\vspace{3in}
}

\author{
	\textbf{\docAuthorNameFirst} \\
	\textbf{\docAuthorNameSecond}
}
\date{} % Insert date here if you want it to appear below your name

%----------------------------------------------------------------------------------------

\begin{document}

\maketitle

%----------------------------------------------------------------------------------------
%	TABLE OF CONTENTS
%----------------------------------------------------------------------------------------

%\setcounter{tocdepth}{1} % Uncomment this line if you don't want subsections listed in the ToC

\newpage
\tableofcontents
\newpage

%----------------------------------------------------------------------------------------
%	PROBLEM 1
%----------------------------------------------------------------------------------------

% To have just one problem per page, simply put a \clearpage after each problem

\begin{docProblem}
\lipsum[1]\vspace{10pt} % Question

\problemAnswer{ % Answer
\lipsum[2]
}
\end{docProblem}

%----------------------------------------------------------------------------------------
%	PROBLEM 2
%----------------------------------------------------------------------------------------

\begin{docProblem}[Exercise \#\arabic{docProblemCounter}] % Custom section title
\lipsum[3] % Question

%--------------------------------------------

\begin{docSection}{(a)} % Section within problem
\lipsum[4]\vspace{10pt} % Question

\problemAnswer{ % Answer
\lipsum[5]
}
\end{docSection}

%--------------------------------------------

\begin{docSection}{(b)} % Section within problem
\problemAnswer{ % Answer
\lipsum[6]
}
\end{docSection}

%--------------------------------------------

\end{docProblem}

%----------------------------------------------------------------------------------------
%	PROBLEM 3
%----------------------------------------------------------------------------------------

\begin{docProblem}[Prob. \Roman{docProblemCounter}] % Roman numerals

%--------------------------------------------

\begin{docSection}{\docProblemName:~(a)} % Using the problem name elsewhere
\problemAnswer{ % Answer
\lipsum[7]
}
\end{docSection}

%--------------------------------------------

\begin{docSection}{\docProblemName:~(b)}
\lipsum[8]\vspace{10pt} % Question

\problemAnswer{ % Answer
\lipsum[9]
}
\end{docSection}

%--------------------------------------------

\end{docProblem}

%----------------------------------------------------------------------------------------
%	PROBLEM 4
%----------------------------------------------------------------------------------------

\begin{docProblem}[Prob. \Roman{docProblemCounter}] % Roman numerals
\problemAnswer{ % Answer
\lipsum[10]
}
\end{docProblem}

%----------------------------------------------------------------------------------------

\end{document}
