%------------------------------------------------------------------------------
%
%------------------------------------------------------------------------------

Cowhub is built using state of the art image recognition and cloud computing technology. However, the key of this project is to provide farmer with an intuitive and easy to use cattle management and identification solution. Hence the conceptualisation of our front ends was human centered.

%------------------------------------------------------------------------------

\begin{subsection}{Persona}
  We designed the following persona to portrait the potential users of Cowhub. Doing so allowed us to extract the demographic characteristics, the needs, the values and lifestyle of farmers.
  \begin{figure}[H]
  	\centering
    \includegraphics[width=1\textwidth]{images/persona.png}
  	\caption[Persona]{Persona - Bob the farmer}
  \end{figure}
\end{subsection}

%------------------------------------------------------------------------------

\begin{subsection}{User Stories}
  From the persona described above we derived the main user stories our app must implement:
  \begin{enumerate}
    \item As a farmer, I want to have a portable cattle registration tool, so I can easily maintain my governmental records.
    \item As a farmer, I want to have a digital catalogue of my herd, so I can manage my cattle.
    \item As a farmer, I want to have an cattle identification tool, so I can recognize a lost cow by simply taking its picture.
  \end{enumerate}
\end{subsection}

%------------------------------------------------------------------------------

\begin{subsection}{Service blueprint}
  Based on the user stories we then conceived our user journeys, see Figures \ref{fig:register-cattle-journey}, \ref{fig:edit-cattle-journey} and \ref{fig:identify-cattle-journey}.
  \begin{sidewaysfigure}
    \includegraphics[width=\textwidth]{images/register-cattle-journey.png}
    \caption{Cattle Registration Journey}
    \label{fig:register-cattle-journey}
  \end{sidewaysfigure}
  \begin{sidewaysfigure}
    \includegraphics[width=\textwidth]{images/edit-cattle-journey.png}
    \caption{Cattle Management Journey}
    \label{fig:edit-cattle-journey}
  \end{sidewaysfigure}
  \begin{sidewaysfigure}
    \includegraphics[width=\textwidth]{images/identify-cattle-journey.png}
    \caption{Cattle Identification Journey}
    \label{fig:identify-cattle-journey}
  \end{sidewaysfigure}
\end{subsection}

%------------------------------------------------------------------------------

\begin{subsection}{User Feedback}
  Most importantly we wanted to have feedback from actual farmers who will potentially use our tool. Therefore we traveled to Jersey and tested the platform on the field. This helped us discovered a few flaws in our application. For instance, when attempting to take a picture of its muzzle, a cow becomes curious and licks the phone. In order to oversome this issue, we decided to implement a zooming feature that enables a user to take the muzzle-shot from a ``safe'' distance. We also realized that sending images requires a robust and stable internet connexion. Hence, we added offline features which preserves a user's herd information locally as to reduce the amount of data it fetches. We might implement later some additional optimizations such as image modification (resizing and displacement) and local request queue.
  
  We also used this opportunity to interview cattle cartakers and breeders to gain some insight about the industry. This actually lead us to think about many things including the potential of this project and its prospects. More on this in the field research section.
\end{subsection}
